\documentclass{scrartcl}

% packages
\usepackage{amsmath}
\usepackage{amssymb}
\usepackage[ngerman]{babel}
\usepackage{booktabs}
\usepackage[font=small,labelfont=bf]{caption}
\usepackage{csquotes}
\usepackage{float}
\usepackage{fontspec}
  \setmainfont[Ligatures=TeX]{Tex Gyre Pagella}
\usepackage{graphicx}
\usepackage[pdfusetitle,unicode]{hyperref}
\usepackage{mathtools}
\usepackage{microtype}
\usepackage{siunitx}
  \sisetup{separate-uncertainty=true}
\usepackage{subcaption}
\usepackage[math-style=ISO,bold-style=ISO]{unicode-math}
  \setmathfont{Tex Gyre Pagella Math}
\usepackage{xfrac}

% options
\setlength{\parindent}{0pt}  % no stupid indentation

% commands
\DeclarePairedDelimiter{\abs}{\lvert}{\rvert}
\DeclarePairedDelimiter{\mean}{\langle}{\rangle}
\renewcommand{\vec}[1]{\mathbf{#1}}
\renewcommand{\i}{\mathrm{i}}
\DeclareRobustCommand{\e}{\ensuremath{\mathrm{e}}}

% meta
\author{Kevin Dungs \and Kevin Heinicke \and Holger Stevens}
\title{Computational Physics}
\subtitle{Übungsblatt 3}

% document
\begin{document}
\maketitle

\section*{Hausaufgabe 7:}


\section*{Hausaufgabe 8: Beliebige Verteilungen erzeugen}
Alle Implementierungen basieren auf Konzepten in C++11. Informationen zum Konzept der Zufallsverteilungen finden sich auf \url{http://en.cppreference.com/w/cpp/concept/RandomNumberDistribution}. Für die Aufgabenteile (a) und (b) wurde das Konzept vollständig implementiert. Bei den anderen Aufgaben wurde aus Zeitgründen darauf verzichtet und nur die Kernfunktionalität umgesetzt.

\paragraph{(a, b)} Die Implementierungen finden sich in \texttt{boxmueller.hpp} und \texttt{centrallimit.hpp}. Mit Hilfe von \texttt{distributions.cc} werden jeweils \num{1e5} Zufallszahlen erzeugt und mit \texttt{hists.py} geplottet. Das Ergebnis ist in \autoref{fig:gaussian} zu sehen. Obwohl die Implementierungen beliebige Mittelwerte und Breiten der Verteilungen erlauben, wird hier die Standardnormalverteilung ($\mu = 0, \sigma = 1$) verwendet.

Für den Mittelwert von $N$ Zufallszahlen $S_N$, deren Verteilung den Erwartungswert $\mathrm{E}[x] = \mu$ und die Varianz $\mathrm{Var}[x] = \sigma^2$ hat, gilt im Limes $N \to \infty$
\begin{equation}
    \sqrt{N}(S_N - \mu) \sim \mathcal{N}(0, \sigma^2)
\end{equation}
wobei $\mathcal{N}$ die Normalverteilung ist. Um also mit Hilfe des zentralen Grenzwertsatzes aus $N$ gleichverteilten Zufallszahlen $x_i$ im Bereich $[0, 1)$ eine Standardnormalverteilung zu approximieren, wird der Mittelwert der $x_i$ berechnet, davon der Mittelwert der Verteilung (hier $\mu = \num{.5}$) abgezogen und das Ergebnis mit $\sqrt{12N}$ multipliziert. Die $12$ ergibt sich aus der Varianz der Gleichverteilung ($\sigma^2 = \sfrac{1}{12}(1 - 0)^2$).

\begin{figure}[H]
    \centering
    \includegraphics[width=.5\textwidth]{plots/gaussian.pdf}
    \caption{Standardnormalverteilungen aus verschiedenen Methoden.}
    \label{fig:gaussian}
\end{figure}

\paragraph{(c)} Die Lösung der Aufgabe ist in \texttt{acceptreject.hpp} zu finden. Auch hier werden mit \texttt{distributions.cc} \num{1e5} Zufallszahlen erzeugt und mit \texttt{hists.py} geplottet. Das Ergebnis ist in \autoref{fig:sinxover2} zu sehen.

\begin{figure}[H]
    \centering
    \includegraphics[width=.5\textwidth]{plots/sinxover2.pdf}
    \caption{Zufallszahlen deren Verteilung $\sin(x)/2$ entspricht.}
    \label{fig:sinxover2}
\end{figure}

\paragraph{(d)} Zunächst muss aus der PDF $f(x) = 3x^2$ die CDF $F(x)$ berechnet werden:
\begin{equation}
    F(x) = \int_0^x\mathrm{d}t\,f(t) = \left[t^3\right]_0^x = x^3
\end{equation}
Als Inverse dieser Funktion ergibt sich
\begin{equation}
    F^{-1}(x) = \sqrt[3]{x}
\end{equation}
Die Implementierung ist in \texttt{transformation.hpp} zu finden. In \autoref{fig:3xsquared} ist das Ergebnis der Erzeugung von \num{1e5} Zufallszahlen zu sehen.

\begin{figure}[H]
    \centering
    \includegraphics[width=.5\textwidth]{plots/3xsquared.pdf}
    \caption{Zufallszahlen deren Verteilung $3x^2$ entspricht.}
    \label{fig:3xsquared}
\end{figure}

\end{document}
