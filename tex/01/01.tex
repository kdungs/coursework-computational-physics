\documentclass{scrartcl}

% packages
\usepackage{amsmath}
\usepackage{amssymb}
\usepackage[ngerman]{babel}
\usepackage{booktabs}
\usepackage[font=small,labelfont=bf]{caption}
\usepackage{csquotes}
\usepackage{float}
\usepackage{fontspec}
  \setmainfont[Ligatures=TeX]{Tex Gyre Pagella}
\usepackage{graphicx}
 \usepackage[pdfusetitle,unicode]{hyperref}
\usepackage{mathtools}
\usepackage{microtype}
\usepackage{siunitx}
  \sisetup{separate-uncertainty=true}
\usepackage{subcaption}
\usepackage[math-style=ISO,bold-style=ISO]{unicode-math}
  %\setmathfont{"[Tex Gyre Pagella Math.ttf]"}
\usepackage{xfrac}

% options
\setlength{\parindent}{0pt}  % no stupid indentation

% commands
\DeclarePairedDelimiter{\abs}{\lvert}{\rvert}
\DeclarePairedDelimiter{\mean}{\langle}{\rangle}
\renewcommand{\vec}[1]{\mathbf{#1}}
\renewcommand{\i}{\mathrm{i}}
\DeclareRobustCommand{\e}{\ensuremath{\mathrm{e}}}

% meta
\author{Kevin Dungs \and Kevin Heinicke}
\title{Computational Physics}
\subtitle{Übungsblatt 1}

% document
\begin{document}
\maketitle

\section*{Hausaufgabe 3: Berechnung von $\mathbf{\pi}$} 

\section*{Hausaufgabe 4: Random-Walk mit Selbstüberschneidung}
Seien $i$ alle möglichen Wege eines Random-Walks gegebener Länge $N$ und $d$ die Dimension, sowie $\vec{e}$ die Einheitsvektoren, folgt für die mittlere Clustergröße eines Random-Walks nach einem Schritt $R_1$:
    \begin{eqnarray*}
        R_1^2 \quad = \quad \langle \vec{r}^2(1) \rangle & = & \sum_i p_i \sum_{k=1}^{2d} \frac{1}{2d}\left(\vec{0} + \vec{e}_k \right)^2 \\
        & = & 1 \\
    \end{eqnarray*}
Der Induktionsschritt liefert die gewünschte Propotionalität:
    \begin{eqnarray*}
        R_{N+1}^2 \quad = \quad \langle \vec{r}^2(N+1) \rangle & = & \sum_i p_i \sum_{k=1}^{2d} \left( \vec{r}_i + \vec{e}_k \right)^2 \\
                                                               & = & \sum_i \frac{p_i}{2d} \left\{ \left( \vec{r}_i + \vec{e}_{x_1} \right)^2 + \left( \vec{r}_i - \vec{e}_{x_1} \right)^2 \right. \\ 
                                                               & & \left. + \dots + \left( \vec{r}_i + \vec{e}_{x_d} \right)^2 + \left( \vec{r}_i - \vec{e}_{x_d} \right)^2 \right\} \\
                                                               & = & \sum_i \left( p_i \vec{r}_i^2 + p_i \right) \\
                                                               & = & \langle \vec{r}^2(N) \rangle + 1 \\
                                                               & \propto & N + 1 \\
        \Rightarrow \quad R_N^2 \quad \propto & N & \propto \quad t
    \end{eqnarray*}


\section*{Hausaufgabe 5: Random-Walk ohne Selbstüberschneidung}


\end{document}
